\documentclass[12pt,twoside]{article} 

\makeatletter
  \let\Title\@title
  \let\Author\@author
\makeatother

\usepackage{fancyhdr}
\renewcommand{\headrulewidth}{0pt}
\raggedbottom

\pagestyle{fancy}
\fancyhf{} % clear all header and footer fields 
%\fancyhead[CE]{\textsc{\hspace*{-1.2in}\Author}}
%\fancyhead[CO]{\textsc{\hspace*{0.9in}\Title}}
%\fancyhead[L]{\textsc{Steve Eichblatt}}
%\fancyfoot[CE]{\hspace*{-1.4in} \Large\selectfont\thepage}
%\fancyfoot[CO]{\hspace*{0.9in} \Large\selectfont\thepage}

\usepackage[letterpaper, portrait, margin=1.0in]{geometry}
\usepackage[utf8]{inputenc}
%\usepackage[english]{babel}
\usepackage[esperanto]{babel}
\usepackage[utf8]{inputenc}
\DeclareUnicodeCharacter{0108}{\^{C}}  % Esperanta litero Ĉ
\DeclareUnicodeCharacter{0109}{\^{c}}  % Esperanta litero ĉ
\DeclareUnicodeCharacter{011C}{\^{G}}  % Esperanta litero Ĝ
\DeclareUnicodeCharacter{011D}{\^{g}}  % Esperanta litero ĝ
\DeclareUnicodeCharacter{0124}{\^{H}}  % Esperanta litero Ĥ
\DeclareUnicodeCharacter{0125}{\^{h}}  % Esperanta litero ĥ
\DeclareUnicodeCharacter{0134}{\^{J}}  % Esperanta litero Ĵ
\DeclareUnicodeCharacter{0135}{\^{\j}}  % Esperanta litero ĵ
\DeclareUnicodeCharacter{015C}{\^{S}}  % Esperanta litero Ŝ
\DeclareUnicodeCharacter{015D}{\^{s}}  % Esperanta litero ŝ
\DeclareUnicodeCharacter{016C}{\u{U}}  % Esperanta litero Ŭ
\DeclareUnicodeCharacter{016D}{\u{u}}  % Esperanta litero ŭ
\usepackage[T1]{fontenc}
\usepackage[urw-garamond]{mathdesign}

\Huge
\renewcommand{\baselinestretch}{1.2}   % space between lines

\title{Similarity Between Esperanto Roots, Based on the Frequency of Words Which Use Them}
\author{Steve Eichblatt}

\begin{document}

\maketitle

\thispagestyle{fancy}
%\large
\section{Introduction}

Esperanto is a very regular language. Words are construction from roots and modifiers. If you imagine a huge matrix of every possible word contructed in this way, those words would all be valid Esperanto words. But the vast majority of them would never be used. Words like \texttt{dislegigon} or \texttt{interlegatiĝi}, with the root word \texttt{leg}, meaning "read",
are senseless words in any language. But substitute the root \texttt{kon}, meaning "know", and the words translate to
"to defamiliarize" and "to get to know one another". 

So the subset of {\it used} words, from the set of all legal words, must say something about human experience. In this report
I will explore this fact, and analyze the Esperanto corpus hoping to find some surprises.

\section{The Data}

I needed 2 data sources for this project, a corpus of used words and also a list of roots. 
I used the downloadable corpus at \texttt{http://tekstaro.com/}, and the online dictionary \texttt{http://reta-vortaro.de/tgz/index.html} for the list of roots. 

The corpus contans about 5 million words, with about 50,000 distinct words. Of course there are significantly fewer roots,
but the corpus doesn't indicate the roots of the words.

\section{Method}

Because the analysis needs the roots of each word, I needed a program to decompose the words into roots and affixes. 
Fortunately due to the regularity of Esperanto, this was an easy program to write.

For example the sentence \texttt{``malfeliĉe la tekstaro ne enhavas la radikojn de ĉiu vorto''} becomes
\texttt{``mal,Feliĉ,e la tekst,ar,o ne en,Hav,as la radik,o,j,n de ĉiu vort,o.''}. Note the root is capitalized 
when it is not the beginning of the word.

The program which did this is relatively simple and not at all perfect. It's biggest shortcoming is that it 
incorrectly decomposes compound words. Thus the word \texttt{``matenmangi''} becomes \texttt{``matenmanĝ,i''}, instead of 
\texttt{``maten,Manĝ,i.''}. Fortunately, compounds words make up a small part of the entire corpus. If someone wanted
to improve the program to handle these cases, it would be straightforward but slow.

The program considers only words which appear more than 3 times in the corpus. Also, if the program decomposes a word 
into roods and affixes, and the affixes appear less than 5 times in the corpus, that word is discarded.

From there the program finds that there are 5,300 valid roots, and 37,000 words in the corpus build from those roots.

Once you have the words decomposed, it is relatively easy to do the analysis to find the similarity between roods.
Any use of a word is considered as a root and affix, and a large table is built.

\section{Analysis}
\subsection{Quantities}

One quickly sees which of the 5,300 roods occur in the most different words. Here are the highest 52 (see table \ref{table:plejmultaj})

Each root becomes 7 words on average. Around 1300 roots participate in only a single word. (eg. \texttt{kvankam, korpulent, ...})
The most often-used roots take part in around 100 words. Clearly the used words are very sparse in the matrix of possible words.

\vspace{.2 in}
\begin{table}[h!]
\begin{center}
\begin{tabular}{ |l|r||l|r||l|r||l|r| }
\hline
radiko& $N_{vortoj}$ & radiko & $N_{vortoj}$ & radiko & $N_{vortoj}$& radiko & $N_{vortoj}$ \\
\hline
ir       &144 & plen      &75 & liber    &   63 &pens         & 59\\
ven      &122 & edz       &72 & sci      &   63 &varm         & 58\\
don      &121 & skrib     &72 & tir      &   63 &rid          & 57\\
kon      &109 & tim       &70 & lev      &   63 &lum          & 56\\
labor    &103 & star      &70 & lig      &   62 &romp         & 56\\
parol     &97 & ten       &69 & proksim  &   62 &uz           & 55\\
am        &96 & met       &68 & fort     &   62 &dorm         & 55\\
est       &90 & trov      &68 & kulp     &   61 &memor        & 55 \\
vid       &88 & prem      &68 & lern     &   61 &dir          & 55\\
mort      &87 & mov       &66 & hav      &   61 &flug         & 54\\
port      &83 & kur       &65 & aper     &   60 &esperant     & 54\\
san       &77 & far       &64 & vetur    &   60 &jun          & 54\\
viv       &77 & ferm      &63 & kompren  &   60 &send         & 54\\ 

\hline
\end{tabular}
\caption{La radikoj uzata en multaj vortojn}\label{table:plejmultaj}
\end{center}
\end{table}
\vspace{.2 in}


What are the 109 words build from the root \texttt{kon}? Here they are from most to least used: \texttt{kon,as kon,at,a kon,is re,Kon,is kon,at,a,j ne,
Kon,at,a kon,i kon,at,iĝ,is re,Kon,i re,Kon,as kon,at,iĝ,i ek,Kon,i kon,at,o,j ne,Kon,at,o kon,ig,is kon,ig,i kon,at,e kon,at,o ek,Kon,is 
ne,Kon,at,a,j kon,o kon,at,a,j,n ne,Kon,at,a,n dis,Kon,ig,i ne,Kon,it,a kon,at,a,n kon,ant,e re,Kon,os ne,Kon,at,ul,o inter,Kon,a 
re,Kon,ebl,a kon,o,j,n re,Kon,u kon,ig,os kon,it,a kon,u kon,o,n kon,at,iĝ,o re,Kon,ant,e kon,at,ig,is re,Kon,o re,Kon,it,a kon,us 
kon,o,j ek,Kon,as re,Kon,ebl,a,j re,Kon,int,e kon,at,iĝ,u kon,ant,o kon,at,o,j,n kon,ig,u kon,at,iĝ,as ne,Kon,at,a,j,n ne,Kon,at,o,n 
ek,Kon,os ne,re,Kon,ebl,a dis,Kon,ig,o re,Kon,us dis,Kon,ig,is re,Kon,at,a kon,at,ig,i re,Kon,o,n ne,Kon,it,a,j kon,ig,as kon,os kon,ig,o 
ek,Kon,o dis,Kon,iĝ,is ne,Kon,at,o,j ek,Kon,u kon,it,a,j kon,at,ec,o kon,int,a re,Kon,int,a ne,Kon,o kon,at,o,n ek,Kon,int,e kon,ig,int,a 
kon,at,ec,o,n kon,iĝ,i inter,Kon,at,iĝ,o kon,at,iĝ,os kon,iĝ,is kon,ant,a kon,ant,a,j ek,Kon,us kon,iĝ,u kon,ig,it,a re,Kon,ebl,as 
dis,Kon,ig,o,n kon,int,e kon,at,in,o kon,at,iĝ,o,n kon,ad,o inter,Kon,at,iĝ,is kon,ant,o,j ne,Kon,it,a,n inter,Kon,at,iĝ,i ne,Kon,it,a,j,n 
kon,iĝ,as ne,Kon,ad,o inter,Kon,a,n ne,Kon,at,ec,o dis,Kon,iĝ,i ne,Kon,ant,a dis,Kon,ig,as kon,ig,ant,e re,Kon,it,a,j dis,Kon,ig,ad,o}.

\subsection{Interrilatoj}

The words organized by roots enables us to estimate the similarity between the roots. Based on the affixes used with any 2 words, and their 
frequency, we can estimate empirically their similarity.
Because the program is too slow to compare every pair of words, I compared only the pairs within the most frequent 300 roots.

Let's compare \texttt{ruĝ} and \texttt{blu}. They seem similar to our minds, no? The program estimates their similarity at around 80\%. 
Why?  Well, around 7\% of the time, \texttt{ruĝ} is is a word with \texttt{iĝ} or \texttt{ig}, for example \texttt{ruĝiĝas, ruĝiĝante}, 
(=\texttt{reddens, reddenning} etc. 
The root \texttt{blu} doesn't "en", (we don't say \texttt{bluen}, although the concept exists). With a bigger corpus we might find words like
\texttt{bluen} but it would not appear very often.

Similarity \texttt{blu} and \texttt{verd} are only 80\% similary. Think \texttt{greenery}.

In fact \texttt{nigr} and \texttt{blank} are much more similar by this measure, about 90\%.

So which are the most similar roots? By this analysis, it is the ``atomic words'' roots which are words in themselves.
For example  \texttt{sed, kvankam, baldaŭ, jes} ( \texttt{but, however, soon, yes}), which always, or nearly always, appear without affixes. 

When we exclude the atomic words from the table, and search for similar words we find that 
\texttt{nokt} (\texttt{night}) and \texttt{vesper} (\texttt{evening}) are the most similar of the roots. 
This is encouraging. Other very similar pairs are \texttt{(knab, vir), (infan, person), (mond, sun), (hor, monat), (hom, person), (lingv, popol),
(famili, popol), (oft, ĝust), (hor, tag), (est, hav), (pov, vol) (histori, lingv)}.
(that is: \texttt{(boy,man), (child, person), (world, sun), (hour, month), (man, person), (language, people), (family, people), (often, correct), (hour, day),
(be, have), (can, want), (history, language)}).

These pairs certainly seem to say something about the human experience.

We would really like to compare all the roots with all of the other roots, but the program is simply too slow. However, we can use 
a method called ``spectral clustering'' to quickly cluster the roots into groups. Then we can estimate the similarity within each cluster. 

\subsection{Root Clusters}
 
I think that the most important criteria for separating roots, which we already saw, is if the root is an atomic word, that is, whether 
there are many different words made from it. So first I separate the roots into 3 groups: The ``vast'' the ``narrow'' and the ``atomic''.
Vast roots take part in more than 7 words (which is more than the average), and never appear as a word without affixes.
Narrow roots take part in 7 words or less, and also never appear as a word without affixes.
Atomic roots sometimes (or always) are words themeselves. Atomic roots must also appear more than 100 times in the corpus.

\subsubsection{Atomic Roots}
Tabelo \ref{table:atomoj} montras la grupoj el la atomaj radikoj per spektra arigato. 

\vspace{.2 in}
\begin{table}[h!]
\begin{center}
\begin{tabular}{ |l||c|r||c|r||c|r||c|r||c|r||r|l| }
\hline
$F$& $m_1$ & $f_1$&$m_2$ & $f_2$&$m_3$ & $f_3$& $m_4$  & $f_4$& $m_5$ & $f_5$& $N_{tot}$ &   plejoftaj radikoj     \\
\hline
1   &--  &98   &e    &.   &a    &.    &o    &.  &ege    &.  &68          &la, kaj, de, en, ne \\
2   &--  &95   &a    &1  &an    &.    &e    &.    &n    &.  &16        &el, per, nur, ĉi, sin \\
3   &--  &90   &a    &2   &e    &2  &ete    &.   &oj    &.  &11      &kun, nun, dum, iom, jen \\
4   &--  &85   &a    &3   &i    &2    &e    &2  &mal    &1   &6  &pli, ĝi, plu, trans, hieraŭ \\
5   &--  &80   &a   &10  &an    &3   &aj    &2    &e    &1   &8         &mi, li, vi, ŝi, apud \\
\hline
\end{tabular}
\caption{La plej grandaj grupoj de atomaj radikoj}\label{table:atomoj}
\end{center}
\end{table}
\vspace{.2 in}

In table \ref{table:atomoj}, $F$ indicates the root family. $m_n$ indicates the $n$-th most common affix used with these roots,
and $f_n$ indicates its frequency. $N_{tot}$ shows the total number of roots in the family. The 5 most often roots appear in the right column.

All of these groups have \texttt{--} as the most common affix. This means that the root is frequently a word.
Group one has the roots \texttt{la,kaj}, (\texttt{the,and}) etc, which nearly always appear unmodified.

The next group in table \ref{table:atomoj} have more and more other modifications. We wee that \texttt{mi, li, vi, ŝi} (\texttt{I/me, he/him, you, she/her})
are together (being 80\% atomic), but \texttt{ĝi} (\texttt{it}) is more often atomic. Note that the pronoun \texttt{si} 
(roughly \texttt{ones self}) is in the narrow roots, so rarely does it appear atomically (around 12\%).

\subsubsection{Narrow Roots}
Table \ref{table:malvastaj} shows the groups from the narrow roots by spectral clustering.
The table shows that the most common affixes (columns $m_1$) are either ``a'' or ``e'' or ``is'' or ``o'' or ``oj''. 
So the roots cluster in word fragments, and most of the word fragments split into more groups.
There are a great number of narrow roots and the table shows only its largest groups. 
Group 5 from the table \ref{table:malvastaj} is the biggest, and the least interesting, containing only names.

The -o affix (basic nouns) is the most common. We see that there are some nouns whose second most common ending is -oj (plural), and 
others whose second most common ending is -on (accusative). We will see this again in the vast roots. 

\vspace{.2 in}
\begin{table}[h!]
\begin{center}
\begin{tabular}{ |l||c|r||c|r||c|r||c|r||c|r||r|l| }
\hline
$F$& $m_1$ & $f_1$&$m_2$ & $f_2$&$m_3$ & $f_3$& $m_4$  & $f_4$& $m_5$ & $f_5$& $N_{tot}$ &   plejoftaj radikoj     \\
\hline
1    &a   &99   &aj    &0  &an    &0  &eco    &0   &oj    &0  &106   &              beat, jid, magr, ajmar, niz \\
2    &a   &36    &e   &16  &aj   &15   &an    &7    &o    &2  &117   &     si, ĝeneral, konstant, sud, eventual \\
3    &e   &36   &aj   &28   &a    &8   &an    &6   &as    &1  &110   &          kelk, subit, precip, plur, nepr \\
4    &is   &16   &as   &10   &i    &9  &ojn    &5  &ado    &4  &382   &            ĉiu, foj, ekzempl, valent, ig \\
5    &o   &99   &on    &0  &oj    &0    &a    &0   &as    &0  &735 & johan, germani, fernand, franci, kristofor \\
6    &o   &80   &on    &6  &oj    &4    &a    &1   &aj    &0  &152 &         faraon, petr, revu, viktor, litovi \\
7    &o   &70   &on   &23  &oj    &1    &a    &0  &ojn    &0  &104 &        situaci, aer, komitat, brust, palac \\
8    &o   &59   &oj   &17  &on   &14  &ojn    &3    &a    &0  &130 &          manier, poet, salon, numer, punkt \\
9    &o   &56   &on   &35  &oj    &1  &ojn    &1    &e    &0  &105 &             mien, plank, frunt, spac, etos \\
10   &o   &53   &on    &7   &a    &6    &e    &4   &aj    &4  &125 &             arme, moskv, pariz, rom, georg \\
11   &o   &43   &on   &25  &oj   &16  &ojn    &6   &is    &1  &105 &          artikol, projekt, task, fraz, aŭt \\
12   &o   &40   &oj   &30  &on   &12  &ojn    &9   &is    &1  &126 &          afer, templ, objekt, figur, event \\
13   &oj   &99  &ojn    &0  &aj    &0   &on    &0    &o    &0  &155 &        juan, pice, nukleotid, flok, gulden \\
\hline
\end{tabular}
\caption{La plej grandaj grupoj de malvastaj radikoj}\label{table:malvastaj}
\end{center}
\end{table}
\vspace{.2 in}

\subsubsection{Vast Roots}
Table \ref{table:vastaj} shows the groups from the vast roots by spectral clustering. 
The vast roots are the richest group to analyze. Its roots are part of more than 7 words so they have many word forms.

\vspace{.2 in}
\begin{table}[h!]
\begin{center}
\begin{tabular}{ |l||c|r||c|r||c|r||c|r||c|r||r|l| }
\hline
$F$& $m_1$ & $f_1$&$m_2$ & $f_2$&$m_3$ & $f_3$& $m_4$  & $f_4$& $m_5$ & $f_5$& $N_{tot}$ &   plejoftaj radikoj     \\
\hline
1     &a   &46  &aj   &15   &an    &8     &e    &5   &ajn    &3   &55 &          ali, sol, propr, angl, sankt \\
2     &a   &31   &e   &30   &aj    &9    &an    &7   &eco    &2   &38 &              bon, tut, sam, long, ĉef \\
3     &a   &27  &aj   &10    &e   &10    &an    &5   &eco    &4   &55 &           grand, nov, bel, plen, grav \\
4     &a   &26   &o   &16    &e   &11    &aj    &9    &an    &5   &34 &         fort, feliĉ, terur, saĝ, real \\
5     &a   &16  &aj    &6    &e    &5  &mala    &4   &eco    &4   &33 &    jun, proksim, supr, liber, interes \\
6  &anto    &5   &o    &4   &oj    &3     &a    &3   &ino    &2   &30 &          naci, san, mov, ofic, kapabl \\
7     &e   &50   &a   &11   &aj    &5     &o    &4    &as    &4   &26 &           mult, ebl, ver, cert, rapid \\
8     &e   &18  &as   &16    &a   &14    &is    &7    &aj    &6   &25 &       sekv, klar, facil, simil, neces \\
9    &is   &32  &as   &18    &i   &11    &os    &3     &u    &3   &92 &               est, dir, pov, hav, dev \\
10    &is   &23  &as   &16    &i   &11     &o   &10    &on    &5   &89 &     rigard, pens, komenc, sent, dezir \\
11    &is   &17  &as   &12    &i   &11   &ado    &3   &ita    &3  &108 &               far, ven, ir, don, trov \\
12    &is   &13  &oj   &12    &o   &10     &i   &10    &as    &9   &46 &         ag, rajt, serv, kant, organiz \\
13    &is    &9   &i    &6   &as    &5   &ita    &4  &isto    &3   &81 &           kon, ferm, rid, ten, instru \\
14     &o   &67  &on   &11   &oj    &3     &a    &3     &e    &1   &21 &         mond, moment, akv, princ, sun \\
15     &o   &55  &oj   &16   &on   &11   &ojn    &3     &a    &2   &28 &           di, dom, voĉ, program, grup \\
16     &o   &51  &on   &19   &oj    &5     &a    &3   &ojn    &2   &40 &          temp, sinjor, urb, kap, part \\
17     &o   &47   &a   &14   &on    &8    &aj    &5     &e    &5   &29 &      vesper, nokt, histori, uson, pac \\
18     &o   &42  &oj   &21   &on   &12   &ojn    &6     &a    &2   &42 &          lingv, tag, libr, ide, popol \\
19     &o   &37  &on   &14  &ino    &4    &oj    &4    &is    &2   &45 &         esperant, viv, patr, fil, ŝip \\
20     &o   &36  &oj   &12    &a   &10    &on    &9    &aj    &4   &30 &         lok, reĝ, ŝtat, famili, flank \\
21     &o   &31   &a   &19    &e   &10    &on    &8    &aj    &8   &40 &      eŭrop, kultur, publik, natur, or \\
22     &o   &26  &oj   &14   &on   &10   &ojn    &6   &ino    &4   &42 &           vir, amik, knab, frat, sign \\
23     &o   &26  &on   &12   &is    &9    &as    &7    &oj    &6   &59 &             labor, nom, edz, tem, lum \\
24     &o   &17   &e   &14   &on    &8     &a    &8    &as    &3   &35 &         fin, ĝoj, silent, hejm, rilat \\
25     &o   &15  &is   &11   &as   &11    &on    &7     &i    &7   &68 &             mort, am, help, daŭr, tim \\
26    &oj   &45   &o   &17  &ojn   &15    &on    &5   &aro    &2   &26 &           hom, jar, vort, okul, membr \\
27    &oj   &32   &o   &30   &on   &10   &ojn    &9     &a    &2   &33 &       land, infan, man, person, pastr \\
28    &oj   &32   &o   &11  &ojn   &11     &a    &4    &on    &4   &18 &       flor, genu, parenc, vers, frukt \\
29    &oj   &23   &a   &22    &o   &14    &aj   &11   &ojn    &5   &15 & scienc, detal, najbar, grek, individu \\
30    &oj   &17   &o   &17   &on    &7   &ojn    &7   &aro    &2   &31 &            verk, arb, vest, kamp, paŝ \\
\hline
\end{tabular}
\caption{La grupoj de vastaj radikoj}\label{table:vastaj}
\end{center}
\end{table}
\vspace{.2 in}

Now we can compare every root in these relatively small groups and see which are the most similar.

In group 2, the ``adverbial adjectives'', the most similare pairs are  \texttt{(brav, naiv), (reciprok, simpl), (brav, strang), (malic, ĉef)
(simpl, sincer), (intim, serioz)} ( \texttt{(brave, naive), (reciprocal, simple), (brave, strange), (malicious, chief) (simple, sincere), (intimate, serious)})
Their similarities are around 85\%.

In group 3, the ``plural adjectives'' the most similar pairs are  \texttt{(dik, grand), (dolĉ, gaj), (dik, mol), (gaj, larĝ), (grav, pez)} 
(\texttt{(fat, bit), (sweet, gay), (fat, soft), (gay, large), (grave, heavy)}). The average similarity between these pairs is about 77\%.

In group 5, the ``opposites adjectives'' the most similar are \texttt{(amuz, interes), (riĉ, spirit), (amuz, distr), (financ, interes)}
(\texttt{(amuse, interest), (rich, spirit), (amuse, distract), (finance, interest)}). Their similarity is only about 55\%.

Group 9, the ``pure verbs'' has the most similar pairs as \texttt{(est, vol), (pov, vol), (est, hav), (ekzist, situ), (ating, konstat)}
(\texttt{(be, want), (can, want), (be, have), (exist, locate), (attain, state)}) Their similarity is about 80\%.

Group 11, the ``ongoing verbs'', has the most similar pairs as: \texttt{(kred, supoz), (kompren, kred), (detru, prepar), (falĉ, plug), (far, trov)}.
(\texttt{(believe, suppose), (understand, believe), (destroy, prepare),(mow, plough), (do, find)}). Their average similarity is about 75\%.

In grupo 13, la ``ist verbs'', the most similar pairs are: \texttt{(pentr, skulpt), (kurac, paŝt), (instru, ĉas), (mok, zorg), (juĝ, ĉas)}
(\texttt{(paint, sculpt), (cure, feed), (instruct, chase), (mock, care for), (judge, chase)}). The similarity here is around 65\%.

In group 17, the ``adjectival nouns'', the most similar pairs are \texttt{(mens, spirit), (afrik, uson),
(printemp, vintr), (nokt, vesper), (printemp, turism)}
(\texttt{(mind, spirit), (africa, USA), (spring (season), winter), (night, evening), (spring (season), tourism)}). 
Similarity withing this group is around 85\%.  

In group 22, the ``feminizable words'', the most similar pairs are: \texttt{(boy, man), (scene, treasure), (argument, reclaim), (paint, interview)}
The similarity here is around 75\%.

Group 26, the ``plural nouns'', the most similar pairs are: \texttt{(soldat, jar), (larm, okul), (branĉ, poem), (poem, vort), (branĉ, foli), (dent, okul)}
(\texttt{(soldier, year), (tear, eye), (branch, poem), (poem, word), (branch, leaf), (tooth, eye)}). Similarity around 85\%.

\section{Conclusion}

Did we learn something about the human experience from this analysis? I am not sure. 
We did succeed in finding similarity between word roots from the forms of the words which use them. I believe that the strict structure
of Esperanto allows us to discover that. I highly doubt that such a simple analysis would be possible in English or any other natural language. 
In this sense, it shows another aspect of the beauty of Zamehof's creation.

\section{Technical Remarks}

This analysis was done using the Python programming language. In you are intersted in extending the analysis, 
see \texttt{https://github.com/eichblatt/analyze\_roots}.
The most important part of this is the word frequency table, a file of about 13 MB, readable in python.

\end{document}
