\documentclass[12pt,twoside]{article} 

\makeatletter
  \let\Title\@title
  \let\Author\@author
\makeatother

\usepackage{fancyhdr}
\renewcommand{\headrulewidth}{0pt}
\raggedbottom

\pagestyle{fancy}
\fancyhf{} % clear all header and footer fields 
%\fancyhead[CE]{\textsc{\hspace*{-1.2in}\Author}}
%\fancyhead[CO]{\textsc{\hspace*{0.9in}\Title}}
%\fancyhead[L]{\textsc{Steve Eichblatt}}
%\fancyfoot[CE]{\hspace*{-1.4in} \Large\selectfont\thepage}
%\fancyfoot[CO]{\hspace*{0.9in} \Large\selectfont\thepage}

\usepackage[letterpaper, portrait, margin=1.0in]{geometry}
\usepackage[utf8]{inputenc}
%\usepackage[english]{babel}
\usepackage[esperanto]{babel}
\usepackage[utf8]{inputenc}
\DeclareUnicodeCharacter{0108}{\^{C}}  % Esperanta litero Ĉ
\DeclareUnicodeCharacter{0109}{\^{c}}  % Esperanta litero ĉ
\DeclareUnicodeCharacter{011C}{\^{G}}  % Esperanta litero Ĝ
\DeclareUnicodeCharacter{011D}{\^{g}}  % Esperanta litero ĝ
\DeclareUnicodeCharacter{0124}{\^{H}}  % Esperanta litero Ĥ
\DeclareUnicodeCharacter{0125}{\^{h}}  % Esperanta litero ĥ
\DeclareUnicodeCharacter{0134}{\^{J}}  % Esperanta litero Ĵ
\DeclareUnicodeCharacter{0135}{\^{\j}}  % Esperanta litero ĵ
\DeclareUnicodeCharacter{015C}{\^{S}}  % Esperanta litero Ŝ
\DeclareUnicodeCharacter{015D}{\^{s}}  % Esperanta litero ŝ
\DeclareUnicodeCharacter{016C}{\u{U}}  % Esperanta litero Ŭ
\DeclareUnicodeCharacter{016D}{\u{u}}  % Esperanta litero ŭ
\usepackage[T1]{fontenc}
\usepackage[urw-garamond]{mathdesign}

\Huge
\renewcommand{\baselinestretch}{1.2}   % space between lines

\title{Similecoj inter Radikoj, Bazita sur la Ofteco de Vortoj kiuj Uzas Ilin}
\author{Steve Eichblatt}

\begin{document}

\maketitle

\thispagestyle{fancy}
%\large
\section{Enkonduko}

Esperanto estas tre regula lingvo. Vortoj estas konstruita el radikoj kaj modifaĵoj. Se oni imagas grandegan tabelon de ĉiu vorto konstuebla, 
t.e. ĉiu radiko kun ĉiu grupo de modifaĵo, ĉiuj vortoj estus laŭreglaj, sed la grandega parto estus neniam uzitaj. 
Vortoj kiel \texttt{dislegigon} aŭ \texttt{interlegatiĝi} ne havas sencon por homoj (Notu: anstataŭigu
la radikon \texttt{leg} kun \texttt{kon}, kaj la du vortoj validiĝas.)

Do, la vortoj uzitaj, el ĉiuj la laŭreglaj ebloj, diras iun pri la homara sperto.  
En ĉi tiu raporto mi volas pripensi tiun fakton, kaj analizi la tekstojn esperante trovi kelkajn surprizojn.

\section{Datumoj}

Mi bezonis 2 fontojn de datumoj por ĉi tiu projekto, tekstaron de vortoj vere uzitaj, kaj ankaŭ liston de radikoj. 
La tekstaron elŝuteblan mi trovis ĉe \texttt{http://tekstaro.com/}. Por la listo de radikoj mi elŝutis la retan vortaron ĉe
\texttt{http://reta-vortaro.de/tgz/index.html}, kaj mi facile tiris liston de radikoj el ĉi tiuj datumoj.

La tekstaro enhavas ĉirkaŭ 5 milionoj da vortoj, ĉirkaŭ 50 mil malsamaj vortoj. Kompreneble, la kvanto da radikoj estas
multe malpli ol 50 mil, sed la vortaro ne indikas la radikojn de la vortoj.

\section{Metodo}

Ĉar la analizo bezonas la radikojn de ĉiu vorto, necesas programon por disigi la vortojn en pecojn. Feliĉe, pro la reguleco
de Esperanto, tia programo estas relative facila por krei. 

Ekzemple, la frazo \texttt{``malfeliĉe la tekstaro ne enhavas la radikojn de ĉiu vorto''} estiĝas 
\texttt{``mal,Feliĉ,e la tekst,ar,o ne en,Hav,as la radik,o,j,n de ĉiu vort,o.''} Notu la radiko estas ĉefliterita
kiam ĝi ne komencigas la vorton. 

La programo kiu faris tion estis relative simpla, kaj tute ne perfekta. Ĝia plej grava manko estas
ke ĝi malkorekte disigas kombinaĵojn. Ekzemple, la vorto \texttt{``matenmangi''} estiĝas \texttt{``matenmanĝ,i''}, anstataŭ
\texttt{``maten,Manĝ,i.''} Feliĉe, la kombinaĵojn konsistigas malgrandan porcion el la tuta tekstaro. Se oni dezirus
plibonigi la programon por trakti tiujn kazojn, mi taksas ke ĝi estus ebla sed malrapidega.

La programo konsideris nur vortojn kiuj aperas pli ol 3 fojoj en la tekstaro. Ankaŭ, se la programo disigis vorton en 
radikon kaj modifaĵon, kaj la modifaĵon aperas malpli ol 5 fojoj en la tuta tekstaro, ĉi tiu vorto estas forjetita.

De tie la programo trovis, ke estas 5,300 apartaj validaj radikoj, kaj 37 mil vortoj en la tekstaro konstruitaj el tiuj radikoj.

Tuj kiam oni havas la vortoj disigitaj, estas relative facila fari la analizon por trovi la similecojn inter radikoj.
Ĉiu uzaĵo de vorto estas konsiderita kun radiko kaj modifaĵoj, kaj granda tabelo estas konstruita.

\section{Analizo}
\subsection{Kvantoj}

Oni povas tuj vidi kiujn el la 5,300 radikoj estas uzata en la plejmulta de vortoj. Jen la plejalta 52 (vidu tabelon \ref{table:plejmultaj})
Ĉiu radiko devenas 7 vortjon, averaĝe. Ĉirkaŭ 1300 radikoj partoprenas en nur unu vorto (ekz. \texttt{kvankam, korpulent, ...})
La plej oftuzitaj radikoj partoprenas en ĉirkaŭ cent vortoj. Do oni vidas ke vortoj estas tre maldensa en la matrico de ebloj.

\vspace{.2 in}
\begin{table}[h!]
\begin{center}
\begin{tabular}{ |l|r||l|r||l|r||l|r| }
\hline
radiko& $N_{vortoj}$ & radiko & $N_{vortoj}$ & radiko & $N_{vortoj}$& radiko & $N_{vortoj}$ \\
\hline
ir       &144 & plen      &75 & liber    &   63 &pens         & 59\\
ven      &122 & edz       &72 & sci      &   63 &varm         & 58\\
don      &121 & skrib     &72 & tir      &   63 &rid          & 57\\
kon      &109 & tim       &70 & lev      &   63 &lum          & 56\\
labor    &103 & star      &70 & lig      &   62 &romp         & 56\\
parol     &97 & ten       &69 & proksim  &   62 &uz           & 55\\
am        &96 & met       &68 & fort     &   62 &dorm         & 55\\
est       &90 & trov      &68 & kulp     &   61 &memor        & 55 \\
vid       &88 & prem      &68 & lern     &   61 &dir          & 55\\
mort      &87 & mov       &66 & hav      &   61 &flug         & 54\\
port      &83 & kur       &65 & aper     &   60 &esperant     & 54\\
san       &77 & far       &64 & vetur    &   60 &jun          & 54\\
viv       &77 & ferm      &63 & kompren  &   60 &send         & 54\\ 

\hline
\end{tabular}
\caption{La radikoj uzata en multaj vortojn}\label{table:plejmultaj}
\end{center}
\end{table}
\vspace{.2 in}


Kio estas la 109 vortoj konstruata el \texttt{kon}? Jen ili vicigata per ofteco: \texttt{kon,as kon,at,a kon,is re,Kon,is kon,at,a,j ne,
Kon,at,a kon,i kon,at,iĝ,is re,Kon,i re,Kon,as kon,at,iĝ,i ek,Kon,i kon,at,o,j ne,Kon,at,o kon,ig,is kon,ig,i kon,at,e kon,at,o ek,Kon,is 
ne,Kon,at,a,j kon,o kon,at,a,j,n ne,Kon,at,a,n dis,Kon,ig,i ne,Kon,it,a kon,at,a,n kon,ant,e re,Kon,os ne,Kon,at,ul,o inter,Kon,a 
re,Kon,ebl,a kon,o,j,n re,Kon,u kon,ig,os kon,it,a kon,u kon,o,n kon,at,iĝ,o re,Kon,ant,e kon,at,ig,is re,Kon,o re,Kon,it,a kon,us 
kon,o,j ek,Kon,as re,Kon,ebl,a,j re,Kon,int,e kon,at,iĝ,u kon,ant,o kon,at,o,j,n kon,ig,u kon,at,iĝ,as ne,Kon,at,a,j,n ne,Kon,at,o,n 
ek,Kon,os ne,re,Kon,ebl,a dis,Kon,ig,o re,Kon,us dis,Kon,ig,is re,Kon,at,a kon,at,ig,i re,Kon,o,n ne,Kon,it,a,j kon,ig,as kon,os kon,ig,o 
ek,Kon,o dis,Kon,iĝ,is ne,Kon,at,o,j ek,Kon,u kon,it,a,j kon,at,ec,o kon,int,a re,Kon,int,a ne,Kon,o kon,at,o,n ek,Kon,int,e kon,ig,int,a 
kon,at,ec,o,n kon,iĝ,i inter,Kon,at,iĝ,o kon,at,iĝ,os kon,iĝ,is kon,ant,a kon,ant,a,j ek,Kon,us kon,iĝ,u kon,ig,it,a re,Kon,ebl,as 
dis,Kon,ig,o,n kon,int,e kon,at,in,o kon,at,iĝ,o,n kon,ad,o inter,Kon,at,iĝ,is kon,ant,o,j ne,Kon,it,a,n inter,Kon,at,iĝ,i ne,Kon,it,a,j,n 
kon,iĝ,as ne,Kon,ad,o inter,Kon,a,n ne,Kon,at,ec,o dis,Kon,iĝ,i ne,Kon,ant,a dis,Kon,ig,as kon,ig,ant,e re,Kon,it,a,j dis,Kon,ig,ad,o}.

\subsection{Interrilatoj}

La vortoj organizita per radikoj ebligas nin taksi la similecon inter radikoj. Bazita sur la modifaĵoj uzita kun la du vortoj, kaj
iliaj oftecoj, oni povas taksi empirie ilian similecon.
Ĉar la program estas tro malrapida por kompari ĉiun vorton kun ĉiu alia vorto, mi komparis nur la 300 plej oftuzatan radikojn.

Ni komparu \texttt{ruĝ} kaj \texttt{blu}. Ili ŝajnas similajn, ĉu ne? La programo taksas ilian simileco 
ĉirkaŭ 80\%. Kial? Nu, po 7\%, ruĝ partoprenas en vorto kun \texttt{iĝ} au \texttt{ig}, ekz. \texttt{ruĝiĝas,ruĝiĝante},k.t.p. 
La radiko \texttt{blu} nek iĝas nek igas. Klare, kun pligranda tekstaro, oni trovus tiujn vortojn, sed ne tre ofte.

Simile, \texttt{blu} kaj \texttt{verd} similas po nur 80\% ankaŭ. Estas \texttt{verdaĵo,verduloj,verdeco}.

Fakte, \texttt{nigr} kaj \texttt{blank} eĉ pli similas, po 90\%. 

Kiuj estas la plej similaj radikoj? Per tiu analizo, estas la ``atomvortoj'', kiuj mem estas vortoj. 
Ekz. \texttt{sed, kvankam, baldaŭ, jes}, k.t.p, kiuj ĉiam, aŭ preskaŭ ĉiam aperas sen modifaĵojn. 

Kiam ni forigi la atomvortojn el la tabelo, kaj serĉi similajn vortojn ni trovas, ke \texttt{nokt}
kaj \texttt{vesper} la plej similaj. Tio estas kuraĝigante. Aliaj tre similaj paroj estas 
\texttt{(knab, vir), (infan, person), (mond, sun), (hor, monat), (hom, person), (lingv, popol), 
(famili, popol), (oft, ĝust), (hor, tag), (est, hav), (pov, vol) (histori, lingv)}. 
Ĉi tiuj paroj certe diras ion pri la sperto homa!

Ni ŝategus kompari ĉiujn radikojn kun ĉiuj aliaj, sed tio estas tro malrapide. Tamen, ni povas uzi 
metodon nomita ``spektra arigato'' (angle: ``spectral clustering'') por rapidege arigi la 
radikojn en malgrandajn arojn. Tiam ni povas taksi la simileco inter la radikoj en ĉiu aro. 

\subsection{Radikaroj}
 
Laŭ mi, la plej grava kriterio por disigi radikoj, kion ni jam vidis, estas se la radiko estas atomvortoj, 
se oni trovas multaj de vortoj konstruita el ĝi. Do, mi unue disigis la radikojn en 3 grupoj: la ``vastaj'',
la ``malvastaj'' kaj la ``atomoj''. Vastaj radikoj faras peco de pli ol 7 vortoj (pli ol la averaĝo), kaj 
ne estas si mem vorto. Malvastaj radikoj faras peco de 7 vortoj aŭ malpli, kaj ankaŭ ne estas si mem vorto.
Atomoj estas iam (aŭ ĉiam) si mem vorto. Atomoj ankaŭ devas aperi pli ol 100 fojo en la tekstaro.

\subsubsection{Atomaj Radikoj}
Tabelo \ref{table:atomoj} montras la grupoj el la atomaj radikoj per spektra arigato. 

\vspace{.2 in}
\begin{table}[h!]
\begin{center}
\begin{tabular}{ |l||c|r||c|r||c|r||c|r||c|r||r|l| }
\hline
$F$& $m_1$ & $f_1$&$m_2$ & $f_2$&$m_3$ & $f_3$& $m_4$  & $f_4$& $m_5$ & $f_5$& $N_{tot}$ &   plejoftaj radikoj     \\
\hline
1   &--  &98   &e    &.   &a    &.    &o    &.  &ege    &.  &68          &la, kaj, de, en, ne \\
2   &--  &95   &a    &1  &an    &.    &e    &.    &n    &.  &16        &el, per, nur, ĉi, sin \\
3   &--  &90   &a    &2   &e    &2  &ete    &.   &oj    &.  &11      &kun, nun, dum, iom, jen \\
4   &--  &85   &a    &3   &i    &2    &e    &2  &mal    &1   &6  &pli, ĝi, plu, trans, hieraŭ \\
5   &--  &80   &a   &10  &an    &3   &aj    &2    &e    &1   &8         &mi, li, vi, ŝi, apud \\
\hline
\end{tabular}
\caption{La plej grandaj grupoj de atomaj radikoj}\label{table:atomoj}
\end{center}
\end{table}
\vspace{.2 in}

En tabelo \ref{table:atomoj}, $F$ indikas la numeron de la familio de radiko. $m_n$ indikas la $n$-a plej oftan vortmodifaĵon, 
kaj $f_n$ indikas ĝian oftecon. $N_{tot}$ montras la tutan kvanton da radikoj en tiu familio. La kvin plej oftaj
radikoj aperas en la plej maldekstra kolumno.

Ĉiuj el tiuj grupoj havas \texttt{--} kiel plej ofta modifaĵo. Tio signifas, ke la radiko mem faras 
vorton. Grupo unu enhavas la radikojn \texttt{la,kaj}, \textit{k.t.p.}, kiuj preskaŭ ĉiam estas senmodifaĵa.
La sekvantaj grupoj en tabelo \ref{table:atomoj} havas pli kaj pli de aliajn modifaĵojn. Ni vidas,
ke \texttt{mi, li, vi, ŝi} estas kune (estanta 80\% atoma), sed \texttt{ĝi} estas pli ofta atoma.  
Notu, ``si'' trovas sin en la malvastaj radikoj, grupo 2; tiel malofte si aperas atome (ĉirkaŭ 12\%).

\subsubsection{Malvastaj Radikoj}
Tabelo \ref{table:malvastaj} montras la grupoj el la malvastaj radikoj per spektra arigato. 
La tabelo montras, ke la plej ofta modifajoj (en kolumno $m_1$) estas aŭ ``a'' aŭ ``e'' aŭ ``is'' aŭ ``o'' aŭ ``oj''.
Do, la radikoj ariĝas en vortopecoj, kaj la plejparto de vortopecoj redividas en plurajn grupojn.
Ni povas vidi, ke estas multegaj malvastaj radikoj, kaj la tabelo montras nur ĝiajn grandajn grupojn.
Grupo 5 el tabelo \ref{table:malvasta} estas la plej granda, kaj la malplej interesa, havanta nur nomoj.

La -o finaĵo estas la plej ofta. Ni vidas, ke estas substantivoj kies dua plej ofta finaĵo estas -oj, kaj 
aliaj kies dua plej ofta finaĵo estas -on. Ni vidos tion ankoraŭ ĉe la vastaj radikoj.

\vspace{.2 in}
\begin{table}[h!]
\begin{center}
\begin{tabular}{ |l||c|r||c|r||c|r||c|r||c|r||r|l| }
\hline
$F$& $m_1$ & $f_1$&$m_2$ & $f_2$&$m_3$ & $f_3$& $m_4$  & $f_4$& $m_5$ & $f_5$& $N_{tot}$ &   plejoftaj radikoj     \\
\hline
1    &a   &99   &aj    &0  &an    &0  &eco    &0   &oj    &0  &106   &              beat, jid, magr, ajmar, niz \\
2    &a   &36    &e   &16  &aj   &15   &an    &7    &o    &2  &117   &     si, ĝeneral, konstant, sud, eventual \\
3    &e   &36   &aj   &28   &a    &8   &an    &6   &as    &1  &110   &          kelk, subit, precip, plur, nepr \\
4    &is   &16   &as   &10   &i    &9  &ojn    &5  &ado    &4  &382   &            ĉiu, foj, ekzempl, valent, ig \\
5    &o   &99   &on    &0  &oj    &0    &a    &0   &as    &0  &735 & johan, germani, fernand, franci, kristofor \\
6    &o   &80   &on    &6  &oj    &4    &a    &1   &aj    &0  &152 &         faraon, petr, revu, viktor, litovi \\
7    &o   &70   &on   &23  &oj    &1    &a    &0  &ojn    &0  &104 &        situaci, aer, komitat, brust, palac \\
8    &o   &59   &oj   &17  &on   &14  &ojn    &3    &a    &0  &130 &          manier, poet, salon, numer, punkt \\
9    &o   &56   &on   &35  &oj    &1  &ojn    &1    &e    &0  &105 &             mien, plank, frunt, spac, etos \\
10   &o   &53   &on    &7   &a    &6    &e    &4   &aj    &4  &125 &             arme, moskv, pariz, rom, georg \\
11   &o   &43   &on   &25  &oj   &16  &ojn    &6   &is    &1  &105 &          artikol, projekt, task, fraz, aŭt \\
12   &o   &40   &oj   &30  &on   &12  &ojn    &9   &is    &1  &126 &          afer, templ, objekt, figur, event \\
13   &oj   &99  &ojn    &0  &aj    &0   &on    &0    &o    &0  &155 &        juan, pice, nukleotid, flok, gulden \\
\hline
\end{tabular}
\caption{La plej grandaj grupoj de malvastaj radikoj}\label{table:malvastaj}
\end{center}
\end{table}
\vspace{.2 in}

\subsubsection{Vastaj Radikoj}
Tabelo \ref{table:vastaj} montras la grupoj el la vastaj radikoj per spektra arigato. 
La vastaj radikoj estas la plej interesa grupo. Ĝia radikoj partoprenas en pli ol 7 vortoj, do ili havas
multajn modifaĵojn. 

\vspace{.2 in}
\begin{table}[h!]
\begin{center}
\begin{tabular}{ |l||c|r||c|r||c|r||c|r||c|r||r|l| }
\hline
$F$& $m_1$ & $f_1$&$m_2$ & $f_2$&$m_3$ & $f_3$& $m_4$  & $f_4$& $m_5$ & $f_5$& $N_{tot}$ &   plejoftaj radikoj     \\
\hline
1     &a   &46  &aj   &15   &an    &8     &e    &5   &ajn    &3   &55 &          ali, sol, propr, angl, sankt \\
2     &a   &31   &e   &30   &aj    &9    &an    &7   &eco    &2   &38 &              bon, tut, sam, long, ĉef \\
3     &a   &27  &aj   &10    &e   &10    &an    &5   &eco    &4   &55 &           grand, nov, bel, plen, grav \\
4     &a   &26   &o   &16    &e   &11    &aj    &9    &an    &5   &34 &         fort, feliĉ, terur, saĝ, real \\
5     &a   &16  &aj    &6    &e    &5  &mala    &4   &eco    &4   &33 &    jun, proksim, supr, liber, interes \\
6  &anto    &5   &o    &4   &oj    &3     &a    &3   &ino    &2   &30 &          naci, san, mov, ofic, kapabl \\
7     &e   &50   &a   &11   &aj    &5     &o    &4    &as    &4   &26 &           mult, ebl, ver, cert, rapid \\
8     &e   &18  &as   &16    &a   &14    &is    &7    &aj    &6   &25 &       sekv, klar, facil, simil, neces \\
9    &is   &32  &as   &18    &i   &11    &os    &3     &u    &3   &92 &               est, dir, pov, hav, dev \\
10    &is   &23  &as   &16    &i   &11     &o   &10    &on    &5   &89 &     rigard, pens, komenc, sent, dezir \\
11    &is   &17  &as   &12    &i   &11   &ado    &3   &ita    &3  &108 &               far, ven, ir, don, trov \\
12    &is   &13  &oj   &12    &o   &10     &i   &10    &as    &9   &46 &         ag, rajt, serv, kant, organiz \\
13    &is    &9   &i    &6   &as    &5   &ita    &4  &isto    &3   &81 &           kon, ferm, rid, ten, instru \\
14     &o   &67  &on   &11   &oj    &3     &a    &3     &e    &1   &21 &         mond, moment, akv, princ, sun \\
15     &o   &55  &oj   &16   &on   &11   &ojn    &3     &a    &2   &28 &           di, dom, voĉ, program, grup \\
16     &o   &51  &on   &19   &oj    &5     &a    &3   &ojn    &2   &40 &          temp, sinjor, urb, kap, part \\
17     &o   &47   &a   &14   &on    &8    &aj    &5     &e    &5   &29 &      vesper, nokt, histori, uson, pac \\
18     &o   &42  &oj   &21   &on   &12   &ojn    &6     &a    &2   &42 &          lingv, tag, libr, ide, popol \\
19     &o   &37  &on   &14  &ino    &4    &oj    &4    &is    &2   &45 &         esperant, viv, patr, fil, ŝip \\
20     &o   &36  &oj   &12    &a   &10    &on    &9    &aj    &4   &30 &         lok, reĝ, ŝtat, famili, flank \\
21     &o   &31   &a   &19    &e   &10    &on    &8    &aj    &8   &40 &      eŭrop, kultur, publik, natur, or \\
22     &o   &26  &oj   &14   &on   &10   &ojn    &6   &ino    &4   &42 &           vir, amik, knab, frat, sign \\
23     &o   &26  &on   &12   &is    &9    &as    &7    &oj    &6   &59 &             labor, nom, edz, tem, lum \\
24     &o   &17   &e   &14   &on    &8     &a    &8    &as    &3   &35 &         fin, ĝoj, silent, hejm, rilat \\
25     &o   &15  &is   &11   &as   &11    &on    &7     &i    &7   &68 &             mort, am, help, daŭr, tim \\
26    &oj   &45   &o   &17  &ojn   &15    &on    &5   &aro    &2   &26 &           hom, jar, vort, okul, membr \\
27    &oj   &32   &o   &30   &on   &10   &ojn    &9     &a    &2   &33 &       land, infan, man, person, pastr \\
28    &oj   &32   &o   &11  &ojn   &11     &a    &4    &on    &4   &18 &       flor, genu, parenc, vers, frukt \\
29    &oj   &23   &a   &22    &o   &14    &aj   &11   &ojn    &5   &15 & scienc, detal, najbar, grek, individu \\
30    &oj   &17   &o   &17   &on    &7   &ojn    &7   &aro    &2   &31 &            verk, arb, vest, kamp, paŝ \\
\hline
\end{tabular}
\caption{La grupoj de vastaj radikoj}\label{table:vastaj}
\end{center}
\end{table}
\vspace{.2 in}

Nun, ni povas kompari ĉiun radikon en tiuj relative malgrandaj grupoj, por vidi la plej similaj.

En grupo 2, la ``adverbaj adjektivoj'', la plej similaj radikparoj estas: \texttt{(brav, naiv), (reciprok, simpl), (brav, strang), (malic, ĉef)
(simpl, sincer), (intim, serioz)}. La simileco inter tiuj paroj ĉirkaŭas 85\%.

En grupo 3, la ``adverbaj adjektivoj'', la plej similaj radikparoj estas: \texttt{(dik, grand), (dolĉ, gaj), (dik, mol), (gaj, larĝ), (grav, pez)}
La simileco inter tiuj paroj ĉirkaŭas 77\%.

En grupo 5, la ``malaj adjektivoj'', la plej similaj radikparoj estas: \texttt{(amuz, interes), (riĉ, spirit), (amuz, distr), (financ, interes)}
La simileco inter tiuj paroj ĉirkaŭas 55\%.

En grupo 9, la ``puraj verboj'', la plej similaj radikparoj estas: \texttt{(est, vol), (pov, vol), (est, hav), (ekzist, situ), (ating, konstat)}
La simileco inter tiuj paroj ĉirkaŭas 80\%.

En grupo 11, la ``ada verboj'', la plej similaj radikparoj estas: \texttt{(kred, supoz), (kompren, kred), (detru, prepar), 
(falĉ, plug), (far, trov)}. La simileco inter tiuj paroj ĉirkaŭas 75\%.

En grupo 13, la ``ista verboj'', la plej similaj radikparoj estas: \texttt{(pentr, skulpt), (kurac, paŝt), (instru, ĉas), (mok, zorg), (juĝ, ĉas)}
La simileco inter tiuj paroj ĉirkaŭas 65\%.

En grupo 17, la ``adjektivaj substantivoj'', la plej similaj radikparoj estas: \texttt{(mens, spirit), (afrik, uson), 
(printemp, vintr), (nokt, vesper), (printemp, turism)}. La simileco inter tiuj paroj ĉirkaŭas 85\%.
 
En grupo 22, la ``ino vortoj'', la plej similaj radikparoj estas: \texttt{(knab, vir), (scen, trezor), (argument, reklam), (farb, intervju)}
La simileco inter tiuj paroj ĉirkaŭas 75\%.

En grupo 26, la ``pluraj substantivoj'', la plej similaj radikparoj estas: \texttt{(soldat, jar), (larm, okul), (branĉ, poem), (poem, vort), (branĉ, foli), (dent, okul)}
La simileco inter tiuj paroj ĉirkaŭas 85\%.

\section{Konkludo}

Ĉu ni lernis ion pri la sperto homa per tio analizo? Mi ne scias. 
Ni ja sukcesis trovi similecoj inter vortradikoj per la formoj de la vortoj kiu uzas ilin. Mi kredas, ke la strikta strukturo de Esperanto permesas
nin malkovri tion. Mi dubegas, ke simila, tiel simpla analizo eblus angle, aŭ alia natura lingvo. 
Tiusence, ĝi montras al ni pluan aspekton de la beleco de Zamenhofa kreaĵo.

\section{Teknika Rimarkoj}

Tiu analizo estis farita uzanta la programlingvo ``python''. Se oni interesas plilabori la analizon, la aŭtoro povas disigi la programon kaj la
datumojn kun vi. La plej grava parto de tio estas la vortkvanto, dosiero je 13 Mbajtoj, legebla uzante python.

\end{document}
